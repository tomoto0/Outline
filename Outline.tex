\documentclass[12pt,halfline,a4paper]{ouparticle}

\usepackage{booktabs}
\usepackage{xcolor}
\usepackage{graphicx}
\usepackage{caption}
\usepackage{subcaption}
\usepackage{cite}  % citeパッケージのインポート
\usepackage{natbib}  % 必要に応じて引用パッケージを使用

\begin{document}

\title{Does the Gender Employment Gap Expand Due to Natural Disasters? Evidence from the Great East Japan Earthquake\break {}}

\author{%
\name{Tomoto Masuda}
\address{}}

\abstract{This study analyzes the impact of the Great East Japan Earthquake of March 2011 on the gender employment gap in Fukushima Prefecture. Employing an event study approach alongside a difference-in-differences (DID) methodology with fixed effects regression, this study examines the earthquake's influence on gender-based employment dynamics. Drawing on extensive prefectural economic statistical data and individual socio-economic attributes from sources such as the Japan General Social Survey (JGSS), the Open Survey Data Japan Household Panel Survey on Consumer Preferences and Satisfaction (JHPS-CPS), the National Census, and the Housing and Land Survey, this study assesses the long-term effects of the earthquake on the gender employment gap. The results indicate that, in contrast to the immediate aftermath, the long-term impacts on the gender employment gap have attenuated significantly. This convergence can be attributed to multiple factors, including persistent governmental initiatives aimed at promoting female labor force participation, the economically rational adaptations of households in response to disruptions in community structures, and the possibility that the destruction of communities by the disaster also disrupted traditional gender norms and local gift economies, pushing women into the labor market.}

\date{\today}

\maketitle

\newpage

\tableofcontents


\newpage

\section{Introduction}
\label{sec1}

How does a natural disaster impact gender disparities within the labor markets of the affected regions? Previous studies present mixed findings on the impact of natural disasters on gender inequality, with some research indicating that such disasters widen gender gaps while other studies suggest a narrowing of these gaps. The Great East Japan Earthquake presented a unique opportunity to examine the dynamics of labor market responses to significant external shocks. This study focuses on Fukushima Prefecture, an area profoundly affected by both the earthquake and subsequent nuclear incident, to analyze the short-term and long-term impacts on the gender employment gap.

To frame the research question, this study aims to present a theoretical household model that encompasses four key elements. First, it considers household decision-making processes, where utility maximization drives choices about labor supply and participation. This model recognizes that households make complex decisions, balancing immediate needs with long-term economic rationality, often leading to gendered patterns of labor market engagement.

Secondly, the model incorporates the concept of economic shocks and adaptation mechanisms. The immediate aftermath of the disaster likely triggered rapid changes in employment patterns. However, over time, households develop adaptation strategies, including adjustments in labor supply preferences, acquisition of new skills, and reallocation of household labor responsibilities.

Thirdly, this study acknowledges the crucial role of government policies in shaping labor market outcomes. The theoretical framework examines how interventions aimed at promoting gender equality in employment can mitigate the adverse effects of economic shocks. This study assesses the effectiveness of these policies in both the short and long term, recognizing that policy impacts may evolve over time.

Lastly, the model considers the impact of community structure on household decision-making. The earthquake and its aftermath significantly disrupted local infrastructure and social networks, potentially altering the context in which households make employment decisions. Additionally, the destruction of traditional gender norms and local gift economies may have forced women into the labor market. I posit that long-term community rebuilding efforts can influence labor market dynamics and potentially reshape gender roles within households and the broader community.

By integrating these four elements - household decision-making, economic shocks and adaptation, policy interventions, and destruction of community - the theoretical model provides a comprehensive framework for understanding the complex interplay of factors affecting the gender employment gap in the wake of a major disaster.

\section{Background}
\label{sec2}

\subsection{The Great East Japan Earthquake }
\label{sec5.1}

 OK, this is ... "In Japan, the triple disaster of the Great East Japan Earthquake, the ensuing tsunami, and the Fukushima Daiichi nuclear crisis will have enduring implications on public perceptions of both local and national government authorities, as well as specialists in relevant fields. Furthermore, this disaster has profoundly reshaped attitudes towards nuclear energy, highlighting its risks and prompting widespread reconsideration of its role in Japan's energy strategy."




The Great East Japan Earthquake of March 2011 resulted in a tripartite catastrophe, comprising a magnitude 9.0 earthquake, a devastating tsunami, and a nuclear accident at the Fukushima Dai-ichi Nuclear Power plant. This disaster precipitated a severe humanitarian crisis, causing extensive damage particularly in the Iwate, Miyagi, and Fukushima prefectures in northeast Japan. According to the National Police Agency, 15,900 people lost their lives and 2,523 people remain unaccounted for, primarily as a result of the
massive tsunami that struck the eastern coast of Japan. The affected
prefectures account for 99.6\% of total fatalities and 99.8\% of total missing persons. In addition, a total of 3,784 fatalities and casualties were recognized as disaster-related deaths in Japan due to the exacerbation of chronic illnesses or suicide during evacuation. Approximately 90\% of fatalities were attributed to drowning. Table 1 presents a summary of the damages in the affected prefectures. The map below indicates the epicenter of the earthquake.


\begin{flushleft}
\begin{table}[h!]
  \caption{Direct Impact Overview in the Disaster-Stricken Prefectures}\label{table:disaster_situation}
  \begin{minipage}[c]{0.4\textwidth}
    \includegraphics[width=\textwidth,height=1.10\textwidth]{epicenter.jpeg}
  \end{minipage}
  \begin{minipage}[c]{0.48\textwidth}
    \raggedright
    \scalebox{0.8}{
    \begin{tabular}{|c|c|c|c|}
    \hline
    & \multicolumn{1}{c|}{Iwate} & \multicolumn{1}{c|}{Miyagi} & \multicolumn{1}{c|}{Fukushima} \\
    \hline
    Population & 1,330,147 & 2,348,165 & 2,029,064 \\
    Deceased & 4,675 & 9,544 & 1,614 \\
    Missing & 1,110 & 1,213 & 196 \\
    Fully destroyed houses & 20,185 & 83,932 & 20,136 \\
    Partially destroyed houses & 4,562 & 138,721 & 65,093 \\
    \hline
    \end{tabular}
    }
  \end{minipage}
\end{table}
\end{flushleft}

Fukushima Prefecture experienced a compound disaster involving both the tsunami triggered by the earthquake and the subsequent nuclear accident. The nuclear incident, in particular, necessitated large-scale evacuations, significantly disrupting local communities and labor markets. This unprecedented situation provides a unique context for examining the long-term socioeconomic impacts of compound disasters.


\subsection{Gender Gap in Japan}
\label{sec5.1}

According to the latest 2024 Global Gender Gap Report by the World Economic Forum (WEF), Japan ranks 118th out of 146 countries, placing it at the bottom among the G7 nations. Particularly, its rankings in the "Economic" and "Political" domains are notably low, with the "Economic" ranking being 120th out of 146 countries. Japan continues to exhibit substantial gender disparities in the economic sector, with the elimination of wage gaps remaining a significant challenge.

The ‘Act on Promotion of Women’s Participation and Advancement in the Workplace’ enacted in April 2016 has led to the development of an environment conducive to working women of child-rearing generations, with initiatives such as the introduction of a reduced working hours system, restrictions on overtime work, and the establishment of childcare facilities within companies. The Gender Equality Bureau of the Cabinet Office has also focused on supporting women in disaster-affected areas by addressing reconstruction efforts from a gender perspective, catering to the specific needs of women, and addressing child-rearing requirements.

\section{Literature review}
\label{sec3}

Development economics models suggest an increase in household members’ labour supply as a shock-coping strategy which is associated with the narrowing of gender employment gaps. Several studies have documented a rise in female labor force participation following natural disasters. For instance, Canessa and Giannelli (2021)\cite{Canessa2021WomensShocks} examine the impact of severe flooding in Bangladesh on women's employment and empowerment using georeferenced and longitudinal household panel data. Their difference-in-differences analysis reveals a significant 13 percentage point increase in women's employment probability post-flood. 


On the other hand, in labor economics, the Risk Adjustment Hypothesis posits that during economic shocks or natural disasters, women are more susceptible to labor market exclusion and face higher risks of deteriorating work conditions or unemployment compared to men. This hypothesis suggests that specific labor market subgroups, particularly female and non-regular workers, often function as 'adjustment valves,' absorbing economic shocks. These groups are disproportionately affected, serving as mechanisms to mitigate broader economic impacts. This framework illuminates the gender-based disparities in employment stability and the uneven distribution of economic resilience across worker categories during crises. For instance, Kim, Ashley, and Corcoran (2014) examines the economic impact of the 2010 earthquake in Haiti, focusing on changes in household composition and employment retention. Authors found that the earthquake caused a significant reduction in employment rates, from 52.6\% prior to the earthquake to 28.6\% five months post-event. Gender disparities were evident, with only 34.2\% of women retaining their employment compared to 55.6\% of men. 


The study tries to fill gaps in the still sparse literature on the impact of natural disasters on gender disparities by differentiating between short-term and long-term effects, addressing the contradictory findings of existing research that suggests either an exacerbation or a reduction of gender gaps.



\section{Data and Descriptive Statistics}
\label{sec4}

To elucidate the causal relationship and mechanisms through which the Great East Japan Earthquake affected the Gender Employment Gap in the disaster-affected regions, this study emphasizes the necessity of analyzing not only regional-level economic statistical panel data but also individual-level microdata pertaining to various socio-economic attributes. The microdata sets employed in this study include:

\subsection{Regional data analysis}
\label{sec4.1}

Figure 1 illustrates the number of new job openings-to-applicants ratio recorded at Hello Work, Japan’s public employment security office, which maintains a comprehensive database of current job vacancies accessible to all citizens. 

\begin{figure}[h!]
    \centering
    \includegraphics[width=0.9\textwidth]{New job openings-to-applicants ratio.png}  % 幅を本文の80%に設定
    \caption{New job openings-to-applicants ratio}
    \label{fig:new_job_openings}
\end{figure}

Although the number of job openings in the three disaster-affected prefectures of Iwate, Miyagi, and Fukushima initially decreased significantly immediately after the disaster, they have since outperformed the national average. This improvement is largely attributed to the reconstruction demand, including a surge in the construction industry. The first research question of this study investigates how the gender employment gap has evolved under such circumstances.

Figure 2 depicts a time-series graph showing the proportion of female job seekers among total job applications submitted to Hello Work in Fukushima Prefecture. A higher proportion suggests a narrowing gender employment gap. 

\begin{figure}[h!]
    \centering
    \includegraphics[width=0.9\textwidth]{Women ratio on the number of applicants in Fukushima.png}  % 幅を本文の80%に設定
    \caption{Women ratio on the number of applicants in Fukushima Pref.}
    \label{fig:women_ratio_fukushima}
\end{figure}

The graph reveals that in Fukushima Prefecture, prior to the disaster, from 2008 to the earthquake, the proportion of male job seekers had increased due to the impact of the Lehman Shock. However, post-disaster, there has been a gradual increase in the proportion of female job seekers. In the disaster-affected areas, there is a potential long-term trend of increasing female labor force participation.

\newpage


Figure 3 presents a line graph illustrating the number of applicants for public unemployment insurance (Employment Insurance) in Fukushima Prefecture, along with the proportion of female applicants relative to total applicants. The graph also includes the national average for comparison. 

\begin{figure}[h!]
    \centering
    \includegraphics[width=0.9\textwidth]{Number of Employment Insurance Decisions_2.png}  % 幅を本文の80%に設定
    \caption{Number of Employment Insurance Decisions by Gender in Fukushima Pref.}
    \label{fig:employment_insurance_decisions}
\end{figure}

From this graph, it is evident that in Fukushima Prefecture, the proportion of female applicants for unemployment insurance has shown a declining trend post-disaster. This suggests a potential improvement in the labor market conditions for female workers in the disaster-affected areas over the longer term.


\subsection{Data sets}
\label{sec5.1}

To elucidate the causal relationship and mechanisms through which the Great East Japan Earthquake affected the Gender Employment Gap in the disaster-affected regions, this study emphasizes the necessity of analyzing not only regional-level economic statistical panel data but also individual-level microdata pertaining to various socio-economic attributes. The microdatasets employed in this study include:

\begin{table}[h]
    \centering
    \caption{Individual-level Surveys}
    \label{tab:annual_income}
    \includegraphics[width=0.9\textwidth]{Statistical surveys.png}  % 幅を本文の80%に設定
\end{table}

\newpage

\subsection{Empirical Strategy}
\label{sec5.1}

This study statistically estimates whether trends similar to those observed at the regional level in microdata can be identified using four anonymized individual-level datasets: the Japan General Social Survey (JGSS), the Open Survey Data Japan Household Panel Survey on Consumer Preferences and Satisfaction (JHPS-CPS), the National Census, and the Housing and Land Survey. The analysis employs a Difference-in-Differences (DID) approach with Fukushima Prefecture as the Treatment group and other prefectures as the Control group.

The empirical strategy of this study employs the difference-in-differences (DID) methodology. The econometric model is specified as follows:
\begin{equation}
Y_{idt} = \alpha_{d} + \beta_{t} + \delta (Treatment_{dt} * Post_{t}) + X_{idt} + \epsilon_{idt}
\end{equation}

Here, \( Y_{idt} \) denotes the outcome variable, such as the employment rate, annual income, or working hours, for individual \( i \) in prefecture \( d \) at year \( t \). To implement the DID analysis, I adopt pooled DID methodology, which involves aggregating data across periods before and after the treatment in cross-sectional data. This method enables the estimation of the average treatment effect across the combined pre-treatment and post-treatment periods.

Using individual-level data from the Japanese General Social Surveys (JGSS), I investigate changes in individual annual income by gender between the three disaster-affected prefectures and other prefectures before and after the earthquake. The JGSS dataset consists of cross-sectional data collected in 2000, 2001, 2002, 2003, 2005, 2006, 2008, 2010, 2012, 2015, 2016, 2017, and 2018, provided by the JGSS Research Center at Osaka University of Commerce. This survey includes responses from over 20,000 individuals aged 20 to 89 who completed self-administered questionnaires. 

Using pooled DID analysis methodology, I present the results in Figure 4, which indicate a statistically significant increase in income among women in the disaster-affected prefectures when comparing the periods before (2000 to 2010) and after (2012 to 2018) the earthquake, over a sufficiently long-term horizon.


\begin{figure}[h!]
    \centering
    \includegraphics[width=0.9\textwidth]{Kernel density graphs of respondent’s annual income from main job.png}  % 幅を本文の80%に設定
    \caption{Kernel density graphs of respondent’s annual income from main job (Income bands, N= 20,119)}
    \label{fig:conceptual_model}
\end{figure}

The pre/post income band difference for affected prefecture women was statistically significant (Table 3; T-value: -1.849, P-value: 0.065). As Figure 4 and Table 3 shows, only this ‘Women in affected prefectures’ group's average income bands shifted right (increased) pre/post-disaster.


\begin{table}[h!]
    \centering
    \caption{Mean of annual income: Pre/Post-disaster period (Income bands, N=20,119)}
    \label{tab:annual_income}
    \includegraphics[width=0.9\textwidth]{Annual income table.png}  % 幅を本文の80%に設定
\end{table}

\newpage

\section{Conclusion}
\label{sec5}

\subsection{Regional level}
\label{sec5.1}

The case of the Great East Japan Earthquake illustrates the complexity of the reconstruction process and rational household responses when communities are disrupted by both tsunami disasters and nuclear accidents. This study presents changes in gender gaps in such complex disaster-affected areas as a conceptual model, as depicted in Figure 5.

\begin{figure}[h!]
    \centering
    \includegraphics[width=0.9\textwidth]{A conceptual model.jpeg}  % 幅を本文の80%に設定
    \caption{A conceptual model of long-term effects on Women's Labor Force Participation}
    \label{fig:conceptual_model}
\end{figure}

Initially, these events disproportionately impacted women workers negatively. However, The earthquake and nuclear disaster potentially accelerated women's labor market participation by fundamentally disrupting communities bound by traditional gender roles. In the short term, the disaster and nuclear accident had a more severe negative impact on female workers. However, in the long term, this catastrophic event, while devastating, may have inadvertently challenged long-standing societal norms, thus facilitating increased female workforce engagement.

\subsection{Household level}
\label{sec5.1}

This study models how households made economically rational decisions following the Great East Japan Earthquake by utilizing a microdata-based DID analysis, which supports the gender employment gaps identified in regional-level statistical data. This household model suggests that the destruction of communities by the earthquake also disrupted traditional gender norms and local gift economies, potentially forcing women into the labor market.

% すべての文献を引用リストに追加
\nocite{*}

% 参考文献リストの場所を示す
\bibliography{references}  % 'references.bib'ファイルの名前を指定
\bibliographystyle{plain}  % 引用スタイルを指定


\end{document}
